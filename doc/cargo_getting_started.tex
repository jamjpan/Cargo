%!TEX root=cargo.tex
\section{Getting Started}

\subsection{Obtaining a Copy}

There are two ways to obtain a copy.

\subsubsection{Release Package}

\subsubsection{Sourcecode}

Clone the master \code{libcargo} git repository to your local computer:
\begin{verbatim}
git clone https://github.com/jamjpan/Cargo.git
\end{verbatim}

The source repository is roughly 78 MB. It is large because it includes problem
instances and road networks. The problem instances together are 40.2 MB, and
the road networks are 34.1 MB.

\subsection{Installing}

The \code{libcargo} library is a \emph{static} library; thus it does not need
to be installed onto your computer. But, it does need to be built in order to
be usable. The release package already contains the pre-built library, but if
you would like to build the library yourself, or if the pre-built version does
not work on your machine, then follow the instructions below to build from source.

\subsection{Building from Source}

Make sure your machine satisfies the prerequisites (Section~\ref{2. Prerequisites}).
Then, \code{cd} into the \code{Cargo} directory and type \code{make}. The
command will create two directories, \code{build} and \code{lib}. Temporary
build objects will be stored in the former, and the library will be in the latter;
its filename will be \code{libcargo.a}.

\subsection{Usage}

Copy the \code{include} directory and \code{libcargo.a} into your own project's root directory. Then
use \code{\#include "libcargo.h"} in your code to access the library. Use
\code{using namespace cargo;} at the top of your code to save prepending the
\code{cargo} namespace to library objects and functions. When building your
project, link \code{libcargo} and all its dependencies. For example, using \code{g++}:
\begin{verbatim}
g++ -std=c++11 -O3 -c -Iinclude -o my_project.o my_project.cpp
g++ my_project.o -lcargo -pthread -lrt -lmetis -ldl
\end{verbatim}
The first \code{g++} command compiles \code{my\_project.cpp} into the
\code{my\_project.o} binary. The \code{-I} flag instructs \code{g++} where to
look for includes; this should match wherever you placed \code{libcargo}'s \code{include}
directory.

