%!TEX root=cargo_reference.tex
\subsection{Classes}

The following classes are defined in \code{include/libcargo/classes.h}:

\subsubsection{Stop}

A \code{Stop} represents a customer or vehicle origin or destination. It is
the basic unit for \emph{schedules}. It is \textbf{immutable}.

\hi{Constructor:} \code{Stop(...)} takes 6 parameters, 1 with default:
\begin{itemize}
    \item[] \code{TripId} -- corresponds to ID of the stop owner
    \item[] \code{NodeId} -- corresponds to the location of the stop
    \item[] \code{StopType} -- type of the stop
    \item[] \code{ErlyTime} -- early time window bound
    \item[] \code{LateTime} -- late time window bound
    \item[] \code{SimlTime v=-1} -- when the stop was visited ($-1=$ not visited)
\end{itemize}

\hi{Example:}
The following creates \code{Stop A\_o} for Customer A's origin.

\code{Stop A\_o(A.id(), A.orig(), StopType::CustOrig, 0, 600);}

\hi{Getters:}
\begin{itemize}
    \item[] \code{owner()} -- returns ID of the stop owner (\code{TripId})
    \item[] \code{loc()} -- returns location of the stop (\code{NodeId})
    \item[] \code{type()} -- returns type of the stop (\code{StopType})
    \item[] \code{early()} -- returns early time window bound (\code{ErlyTime})
    \item[] \code{late()} -- returns late time window bound (\code{LateTime})
    \item[] \code{visitedAt()} -- returns when the stop was visited (\code{SimlTime})
\end{itemize}

\hi{Equality Comparator:}

Two \code{Stop}s are equal if their owners and locations are equal.

\subsubsection{Schedule}

A \code{Schedule} is a sequence of stops that vehicles visit in order.
Schedules are associated with a particular vehicle. YOU MUST ensure a vehicle's
schedule and its route are synchronized (\emph{i.e.} all stops in the schedule
are visited by the route).  Schedules are \textbf{immutable}. To change a
vehicle's existing schedule, create a new one and commit it into the database
using method \code{assign} (\code{rsalgorithm.h}).

\hi{Constructor:} \code{Schedule(...)} takes 2 parameters:
\begin{itemize}
    \item[] \code{VehlId} -- corresponds to ID of the schedule owner
    \item[] \code{vec\_t<Stop>} -- the raw sequence of \code{Stop}s
\end{itemize}

\hi{Example:}
The following creates schedule \code{sch} for vehicle \code{I} traveling from
its origin, then to pick up customer \code{A} at \code{A\_o}, then to
drop off the customer at \code{A\_d}, then to arrive at \code{I\_d}.
\code{\\ vec\_t<Stop> I\_sch \{I\_o, A\_o, A\_d, I\_d\}; \\ Schedule sch(I.id(), I\_sch);}

\hi{Getters:}
\begin{itemize}
    \item[] \code{owner()} -- returns ID of the schedule owner (\code{VehlId})
    \item[] \code{data()} -- returns raw sequence of stops (\code{vec\_t<Stop>})
    \item[] \code{at(SchIdx)} -- returns stop at index \code{SchIdx} (\code{Stop})
    \item[] \code{front()} -- returns first stop in the schedule (\code{Stop})
    \item[] \code{size()} -- returns number of stops (\code{size\_t})
    \item[] \code{print()} -- prints schedule to standard out (\code{void})
\end{itemize}

\subsubsection{Route}

A \code{Route} is a sequence of waypoints that a vehicle travels along.  Routes
are associated with a particular vehicle. Routes are \textbf{immutable}.  The
function \code{route\_through(...)} (\code{functions.h}) creates a shortest
route given a schedule.

\hi{Constructor:} \code{Route(...)} takes 2 parameters:
\begin{itemize}
    \item[] \code{VehlId} -- corresponds to ID of the schedule owner
    \item[] \code{vec\_t<Wayp>} -- the raw sequence of \code{Wayp}s
\end{itemize}

\hi{Example:} The following creates route \code{rte} using the schedule in the
above example for vehicle \code{I}.
\code{\\ vec\_t<Wayp> I\_rte; \\ route\_through(I\_sch, I\_rte); \\ Route rte(I.id(), I\_rte);}

\hi{Getters:}
\begin{itemize}
    \item[] \code{owner()} -- returns ID of the route owner (\code{VehlId})
    \item[] \code{data()} -- returns raw sequence of waypoints (\code{vec\_t<Wayp>})
    \item[] \code{node\_at(RteIdx)} -- returns ID of node at index \code{RteIdx} (\code{NodeId})
    \item[] \code{dist\_at(RteIdx)} -- returns distance to node at index (\code{DistInt})
    \item[] \code{cost()} -- returns total distance remaining (\code{DistInt})
    \item[] \code{at(RteIdx)} -- returns waypoint at index \code{RteIdx} (\code{Wayp})
    \item[] \code{size()} -- returns number of waypoints (\code{size\_t})
    \item[] \code{print()} -- prints schedule to standard out (\code{void})
\end{itemize}

\subsubsection{Trip}

Class \code{Trip} is the base class representing customers and vehicles. It is
\textbf{immutable}.

\hi{Constructor:} \code{Trip(...)} takes 6 parameters:
\begin{itemize}
    \item[] \code{TripId} -- corresponds to ID of the trip
    \item[] \code{OrigId} -- location of the trip origin
    \item[] \code{DestId} -- location of the trip destination
    \item[] \code{ErlyTime} -- early time window bound
    \item[] \code{LateTime} -- late time window bound
    \item[] \code{Load} -- load of the trip; negative indicates capacity
\end{itemize}

\hi{Getters:}
\begin{itemize}
    \item[] \code{id()} -- returns ID of the trip (\code{TripId})
    \item[] \code{orig()} -- returns location of the origin (\code{OrigId})
    \item[] \code{dest()} -- returns location of the destination (\code{DestId})
    \item[] \code{early()} -- returns early time window bound (\code{ErlyTime})
    \item[] \code{late()} -- returns late time window bound (\code{LateTime})
    \item[] \code{load()} -- returns load of the trip (\code{Load})
\end{itemize}

\subsubsection{Customer}

A \code{Customer} represents a ridesharing customer. Usually it is not
constructed by the user. It is \textbf{immutable}.

\hi{Constructor:} \code{Customer(...)} takes 8 parameters, with 1 default parameter:
\begin{itemize}
    \item[] \code{CustId} -- corresponds to ID of the customer
    \item[] \code{OrigId} -- location of the origin
    \item[] \code{DestId} -- location of the destination
    \item[] \code{ErlyTime} -- early time window bound
    \item[] \code{LateTime} -- late time window bound
    \item[] \code{Load} -- load of the customer; should be positive
    \item[] \code{CustStatus} -- customer status
    \item[] \code{VehlId a=-1} -- assigned to which vehicle ($-1$= unassigned)
\end{itemize}

\hi{Getters:} In addition to getters for \code{Trip}, \code{Customer} has the
following getters:
\begin{itemize}
    \item[] \code{status()} -- returns status of the customer (\code{CustStatus})
    \item[] \code{assignedTo()} -- returns vehicle customer is assigned to (\code{VehlId})
    \item[] \code{assigned()} -- returns true if customer is assigned (\code{bool})
    \item[] \code{print()} -- prints customer to standard out (\code{void})
\end{itemize}

\hi{Equality Comparator:}

Two \code{Customer}s are equal if their IDs are equal.

\subsubsection{Vehicle}

A \code{Vehicle} represents a ridesharing vehicle. Usually it is not
constructed by the user. It is \textbf{immutable}.

\hi{Constructor:} \code{Vehicle(...)} has two constructors. The first takes
7 parameters:
\begin{itemize}
    \item[] \code{VehlId} -- corresponds to ID of the vehicle
    \item[] \code{OrigId} -- location of the origin
    \item[] \code{DestId} -- location of the destination
    \item[] \code{ErlyTime} -- early time window bound
    \item[] \code{LateTime} -- late time window bound
    \item[] \code{Load} -- load of the vehicle; should be negative
    \item[] \code{G\_Tree\&} -- which g-tree index to use for constructing
        vehicle's default route
\end{itemize}
\hi{} The second constructor takes 12 parameters. The first 6 are the same
as the top construct. The following 6 are:
\begin{itemize}
    \item[] \code{Load} -- number of customers queued to be served
    \item[] \code{DistInt} -- distance to next node in its route
    \item[] \code{Route} -- specific route for the vehicle
    \item[] \code{Schedule} -- specific schedule for the vehicle
    \item[] \code{RteIdx} -- index of vehicle's last-visited node in its route
    \item[] \code{VehlStatus} -- vehicle status
\end{itemize}

\hi{Getters:} In addition to getters for \code{Trip}, \code{Vehicle} has the
following getters:
\begin{itemize}
    \item[] \code{next\_node\_distance()} -- returns distance to next node in the route (\code{DistInt})
    \item[] \code{route()} -- returns route (\code{Route})
    \item[] \code{idx\_last\_visited\_node()} -- returns index of vehicle's last-visited node in its route (\code{RteIdx})
    \item[] \code{last\_visited\_node()} -- returns last-visited node (\code{NodeId})
    \item[] \code{status()} -- returns status (\code{VehlStatus})
    \item[] \code{queued()} -- returns number of customers queued to be served (\code{Load})
    \item[] \code{capacity()} -- returns maximum capacity (\code{Load})
    \item[] \code{print()} -- prints vehicle to standard out (\code{void})
\end{itemize}

\hi{Equality Comparator:}

Two \code{Vehicles}s are equal if their IDs are equal.

\subsubsection{MutableVehicle}

A \code{MutableVehicle} is a \code{Vehicle} that can be modified.

\hi{Constructor:} \code{MutableVehicle(...)} is constructed via copy-constructor:
\begin{itemize}
    \item[] \code{Vehicle\&} -- create a \code{MutableVehicle} version of \code{Vehicle}
\end{itemize}

\hi{Methods:} In addition to getters for \code{Vehicle}, \code{MutableVehicle} has the
following methods:
\begin{itemize}
    \item[] \code{set\_rte(const vec\_t<Wayp>\&)} -- set new route (\code{void})
    \item[] \code{set\_rte(const Route\&)} -- set new route (\code{void})
    \item[] \code{set\_sch(const vec\_t<Stop>\&)} -- set new schedule (\code{void})
    \item[] \code{set\_sch(const Schedule\&)} -- set new schedule (\code{void})
    \item[] \code{set\_nnd(const DistInt\&)} -- set next-node distance (\code{void})
    \item[] \code{set\_lvn(const RteIdx\&)} -- set index of last-visited node in route (\code{void})
    \item[] \code{reset\_lvn()} -- set index of last-visited node to 0 (\code{void})
    \item[] \code{incr\_queued()} -- increase by 1 \# of customers queued to be served (\code{void})
    \item[] \code{decr\_queued()} -- decrease by 1 \# of customers queued to be served (\code{void})
\end{itemize}

\subsubsection{ProblemSet}

A \code{ProblemSet} represents a ridesharing problem instance. Usually it is
not constructed by the user.

\hi{Constructor:} \code{ProblemSet()} has an empty constructor.

\hi{Methods:}
\begin{itemize}
\item[] \code{name()} -- returns name of the instance (\code{std::string\&})
\item[] \code{road\_network()} -- returns name of road network (\code{std::string\&})
\item[] \code{set\_trips(const dict<ErlyTime, vec\_t<Trip>>\&)} -- store trips (\code{void})
\end{itemize}


